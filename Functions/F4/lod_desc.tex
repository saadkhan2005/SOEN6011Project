\documentclass{article}
% \usepackage{arxiv}
	\addtolength{\oddsidemargin}{-.875in}
	\addtolength{\evensidemargin}{-.875in}
		\addtolength{\textwidth}{1.75in}

	\addtolength{\topmargin}{-.875in}
	\addtolength{\textheight}{1.75in}
\title{\textbf {Logarithm Description}}



\begin{document}
\maketitle

% keywords can be removed
\section{Introduction}
The logarithm in mathematics is the reverse or inverse of another function. We can find logarithm of a number ‘x’ that is it is the exponent of the number ‘b’ which produces ‘x’. For instance if we take Log$_{b}(x)$ then the "logarithm to base 10" of 1000 is 3. So here ‘b’ is base. When b=10 it is a common logarithm. In programming we can use logarithmic exponents because it simplifies complex mathematical calculations. Computer developers use logarithms in computer function formulas to create specific software program outcomes, such as the creation of graphs that compare statistical data.


\begin{equation}
Log_{10}(1000)=3
\end{equation}
\section{Domain and Range}
The domain of the logarithmic function y=logbx is the set of positive real numbers and the range is the set of real numbers.
Domain : $(0,\infty)$
Range:  $(-\infty,\infty)$
\section{Rules}
Logarithms are commonplace in scientific formulae, and in measurements of the complexity of algorithms and of geometric objects called fractals. 
\begin{itemize}
\item	Product:			log$_{b}(xy)  = log_{b}(x)+ log_{b}$(y)
\item	Ratios:			log$_{b}(x/y) = log_{b}(x)- log_{b}$(y)
\item	Power:			log$_{b}(xy)   = y log_{b}$(x)
\end{itemize}
\section{References}

 [1]"Logarithm", En.wikipedia.org, 2019. [Online]. Available: https://en.wikipedia.org/wiki/Logarithm. [Accessed: 06- Jul- 2019].\\
 [2] Techwalla, 2019. [Online]. Available: https://www.techwalla.com/articles/uses-of-logarithms-in-computers. [Accessed: 06- Jul- 2019].


\end{document}
