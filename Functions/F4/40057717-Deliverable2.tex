\documentclass{article}

	\addtolength{\oddsidemargin}{-.875in}
	\addtolength{\evensidemargin}{-.875in}
		\addtolength{\textwidth}{1.75in}

	\addtolength{\topmargin}{-.875in}
	\addtolength{\textheight}{1.75in}
\title{\textbf {Logarithm Function}}



\begin{document}
\maketitle

\section{Eclipse Debugger}
\subsection{Description}

Debugging is the routine process of locating and removing bugs, errors or abnormalities from programs. Debugging helps to check and watch the variables and source code of the program. In the source code, we have to use breakpoints where the execution of the program should stop during debugging so that we can identify the bugs. In this mode, we can specify watchpoints.


\subsection{Debugging support in Eclipse}
We can start a program in Debug mode in Eclipse. I used eclipse built-in debugger in the project so that breakpoints can be inserted in any line of statement to check the ability or disability of the program.
\subsection{Advantages}
\begin{itemize}
\item Eclipse provides buttons in the toolbar for controlling the execution of the program you are debugging. Typically, it is easier to use the corresponding keys to control this execution.
\item The values of variables can be viewed from the current stack using Variables view.
\item The Variables view allows you to change the values assigned to your variable at runtime
\item We can move the current execution while executing (drag the arrow that indicates the current line)
\item We can step into code/out of/over code based on whether or not you think it matters
\end{itemize}

\subsection{Disdvantages}
\begin{itemize}

\item When you debug your application you probably use the Variables view. The disadvantage is that if it is a complex object then we cannot see the data inside it unless we implement toString() method with that.
\end{itemize}





\begin{document}
\maketitle
\newpage 
% keywords can be removed
\section{Quality attributes}
There are some quality attributes that has been achieved and the efforts that has been made:
\subsection{Correctness}
This system is functionally correct because it behaves according to its functional requirements and specifications. Assessments have been done using Testing and Verification so that the program maintains its correctness.
\subsection{Maintainable}
This system is perfective and adaptive. There are different methods for different functionality so that the changes can be made easily in the future.


\subsection{Efficiency}
For efficiency, I have used recursive function which produces faster and with higher quality result.

\subsection{Usability}
This system provides preventative error handling for user. It prompts message for complex operation.
\subsection{Robust}
In this system when the input is not specified then the system handles the response should be correct.
\newpage
\section{Quality of Source Code}
\subsection{Checkstyle}
The program uses checkstyle Plugin-in 8.1 8.0 for code analysis. Checkstyle is a static code analysis tool used in software development for checking if Java source code complies with coding rules. It helped me to make indentation level correct, remove extra spaces in the program and make the coding styles as per rules. It examined the following in my project:

\begin{itemize}

\item	Javadoc comments for classes, attributes and methods.
\item	Set naming conventions of variables and methods.
\item	The number of parameters in functions.
\item	Lengths of line.
\item	The presence of mandatory headers.
\item The use of imports, and scope modifiers.
\item	The spaces between some characters.
\item	The practices of class construction.
\item	Multiple complexity measurements.

\end{itemize}



\subsection{Adnvantages}
The programming style adopted by a software development project can help to ensure that the code complies with good programming practices which improves the quality, readability, re-usability of the code and may reduce the cost of development. 

\subsection{Disadvantages}
The checks performed by Checkstyle are mainly limited to the presentation of the code. These checks do not confirm the correctness or completeness of the code.

\section{References}
\begin{itemize}
 \item"Logarithm", En.wikipedia.org, 2019. [Online]. Available: https://en.wikipedia.org/wiki/Logarithm. [Accessed: 06- Jul- 2019].

\item En.wikipedia.org. (2019). Checkstyle. [online] Available at: https://en.wikipedia.org/wiki/Checkstyle [Accessed 28 Jul. 2019].

\item Lars Vogel (c) 2009, 2. (2019). Java Debugging with Eclipse - Tutorial. [online] Vogella.com. Available at: https://www.vogella.com/tutorials/EclipseDebugging/article.html [Accessed 29 Jul. 2019]. 
\item GitHub. (2019). saadkhan2005/SOEN6011Project. [online] Available at:$\\$ https://github.com/saadkhan2005/SOEN6011Project [Accessed 29 Jul. 2019].
\end{itemize}

\end{document}

\end{document}
